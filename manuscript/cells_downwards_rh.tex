\documentclass{article}
\usepackage{amsmath,amssymb,amsthm}

\newtheorem{theorem}{Theorem}
\newtheorem{lemma}{Lemma}
\newtheorem{definition}{Definition}

\begin{document}

\title{Cells Downwards and Spectral Rigidity toward RH}
\author{}
\date{}
\maketitle

\section{Spectral Setup}

Let $\xi(s)$ denote the completed Riemann xi-function.
It satisfies the functional equation
\[
\xi(s) = \xi(1-s).
\]

Define the Hilbert space
\[
\mathcal{H} = L^2\left(\mathbb{R}, w(t)\,dt\right)
\]
for an appropriate weight $w$ induced by the Mellin transform of $\xi$.

\section{Cell Decomposition}

We decompose $\mathcal{H}$ into spectral cells
\[
\mathcal{H} = \bigoplus_{k} \mathcal{C}_k
\]
indexed by imaginary height bands.

\section{Downward Transfer Operator}

Define a transfer operator $T$ mapping cells downward in height.

The key obstruction statement is:

\begin{lemma}[Downward Rigidity Target]
If a zero of $\xi(s)$ satisfies $\Re(s) \neq \tfrac{1}{2}$,
then there exists a non-decaying mode in some cell $\mathcal{C}_k$
that is invariant under $T$.
\end{lemma}

\section{Missing Inequality}

To complete the proof it suffices to prove a coercive estimate of the form
\[
\|T f\|^2 \leq \|f\|^2 - \delta \|f_{\perp}\|^2
\]
for some $\delta > 0$ unless $f$ is symmetric under $s \mapsto 1-s$.

\end{document}
\section{Next Step Placeholder}
egin{theorem}Placeholder.nd{theorem}

\section{Critical Symmetry Decomposition}

Define the involution $J$ by
\[
(Jf)(s) := f(1-s).
\]

Then $J^2 = I$ and $\mathcal{H}$ decomposes as
\[
\mathcal{H} = \mathcal{H}_+ \oplus \mathcal{H}_-
\]
where
\[
\mathcal{H}_\pm = \{ f \in \mathcal{H} : Jf = \pm f \}.
\]

The Riemann Hypothesis is equivalent to the absence of
non-decaying modes in $\mathcal{H}_-$.

\section{Quadratic Form Model}

Define a quadratic form
\[
Q(f) := \langle L f, f \rangle
\]
where $L$ is a self-adjoint operator on $\mathcal{H}$.

We assume:

1. $L$ commutes with the symmetry $J$.
2. $L$ is positive on $\mathcal{H}_+$.
3. $L$ has no kernel except possibly in $\mathcal{H}_+$.

The Riemann Hypothesis is equivalent to:
\[
Q(f) > 0 \quad \text{for all } f \in \mathcal{H}_- \setminus \{0\}.
\]

\end{document}

\section{Weil Explicit Quadratic Form}

Let $\phi$ be a test function with Mellin transform $\widehat{\phi}$.

Define the quadratic form
\[
Q(\phi) :=
\sum_\rho \widehat{\phi}(\rho)\widehat{\phi}(1-\rho)
- \int_{0}^{\infty} \Phi(x)\, dx,
\]
where the sum runs over nontrivial zeros $\rho$ of $\zeta(s)$
and $\Phi$ is the prime-weight contribution from the explicit formula.

The Riemann Hypothesis is equivalent to:

\[
Q(\phi) \ge 0 \quad \text{for all admissible } \phi.
\]

\end{document}

\section{Explicit Formula Contributions}

\subsection{Prime (Mangoldt) Term}

Let $\Lambda(n)$ denote the von Mangoldt function.
For admissible $\phi(x)=g(\log x)$ define
\[
\Phi_{\mathrm{pr}}(g)
:= 2\sum_{n\ge 2} \frac{\Lambda(n)}{\sqrt{n}}\, g(\log n).
\]

\subsection{Gamma Term}

Define the archimedean contribution
\[
\Phi_{\Gamma}(g)
:= \int_{-\infty}^{\infty} g(t)\, W_{\Gamma}(t)\, dt,
\]
where $W_{\Gamma}$ is the even weight induced by the
logarithmic derivative of the $\Gamma$-factor of $\xi$.

\subsection{Completed Quadratic Form}

\[
Q(g)
:= \sum_{\rho} \widehat{\phi}(\rho)\widehat{\phi}(1-\rho)
- \Phi_{\mathrm{pr}}(g)
- \Phi_{\Gamma}(g).
\]

We seek:

\[
Q(g) \ge \delta \|g_-\|^2
\quad (\exists \delta>0).
\]

